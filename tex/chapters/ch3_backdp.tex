\chapter{Digital pathology}
\label{chap:backdp}

\begin{overview}{Overview}
  The goal of this chapter is to provide digital pathology background  and keys to understand our contributions. We will focus our attention on topics relevant to this thesis. 
  
  Section \ref{sec:backdp:whatisdp} introduces and defines medical terms such as \textit{pathology}, defines \acrfirstit{digital pathology} and introduces the notion of \acrfirstit{wsi}. Section \ref{sec:backdp:wsi} presents the journey of a sample from the body to the \acrshort{wsi}, introducing the different sources of variability introduced by the whole conversion process. 
\end{overview}

% analogintelligence.com image dp illustration

\section{What is digital pathology?}
\label{sec:backdp:whatisdp}

Nowadays, medicine and healthcare rely heavily on analysis of human body samples to study and diagnose diseases. The branch of medicine focusing on this analysis is called \textit{pathology} which includes histology-based pathology (\aka histopathology) or cytology-based pathology (\aka cytopathology). Both of these sub-branches involve the study of microscope glass slides containing samples (see Figure \ref{fig:backdp:glassslides}). In the case of histology, these samples are tissue sections cut from a bodily specimen. Cytology, on the other hand, is concerned with samples of free cells or tissue fragments which can be extracted by different techniques. 

\begin{figure}
  \centering
  \includegraphics[scale=0.75]{backdp/microscope-slide.png}
  \caption{Microscope slides with tissue samples (image from \parencite{img:glassslides}).}
  \label{fig:backdp:glassslides}
\end{figure}

The trend of digitalization affecting our societies also impacts pathology as, using dedicated scanners, these glass slides can now be digitized into image files called \acrfirstit{wsi}. In this context, \acrfirstit{dp} can be defined as ``\textit{the acquisition, management, sharing and interpretation of pathology information — including slides and data — in a digital environment}'' \parencite{doolan2019whatisdp}. Working with \acrshort{wsi} instead of physical slides has several advantages and drawbacks (see Table 1 in \parencite{jahn2020digital}). The digital format allows for easy sharing and improves sustainability of slides. Glass slides are harder to share across different services or hospitals, can be damaged or lost and degrade over time. Digitization also enables the use of new visualization tools for review and diagnosis. More importantly in the context of this thesis, it opens the way for automated analysis sofware to extract relevant information for pathologists.

\section{Whole-slide images: a journey from the body to the computer}
\label{sec:backdp:wsi}

\parencite{mccann2014automated}

\subsection{Biopsy and smearing}
\subsection{Staining}
\subsection{Scanners}

\section{Professions and typical tasks}
\label{sec:backdp:professionandtasks}

\section{Machine learning}
\label{sec:backdp:ml}

\subsection{Data leakage}
\label{ssec:backdp:dataleakage}

\subsection{Data scarcity}
\label{ssec:backdp:datascarcity}
% and imperfect annotations

\subsection{Transfer learning}
\label{ssec:backdp:tl}

\parencite{van2019strategies}

\section{Visualization and analysis tools}

\subsection{Cytomine}

\subsection{Others}

\section{Digital pathology datasets}
\label{sec:backdp:dataset}


% origin, acknowledgements, subject, organ, staining, statistics


\subsection{Thyroid nodule fine-needle aspiration biopsy}



\subsection{Publicly available}