\chapter{Machine learning}
\label{chap:backml}

This thesis focuses on the application of machine learning to digital pathology (see Chapter \ref{chap:backdp}) problems. Therefore this chapter aims at giving to a reader with basic knowledge of \acrshort{ml} the keys to understand our contributions. It is not aimed at being a thorough presentation of the different methods but rather an overview. For the readers who would still like to deepen their knowledge about these methods, we will provide pointers to relevant litterature.  

\section{What is machine learning ?} 
\label{sec:backml:whatisml}

\acrlong{ml} is a subfield of \acrlong{ai} which focuses on methods enabling computers to learn from data. More formally, a machine learning algorithm is defined by a set of candidate models (\ie an hypothesis space $\mathfrak{H}$), a quality measure for a model and an optimization strategy. The algorithm output is a model $h \in \mathfrak{H}$ that maximises the quality criterion 

\subsection{Families of learning methods}

There are many ways to structure the ecosystem of machine learning methods.  


% ===================
% Eval and selection
% ===================
\section{Model evaluation and selection}

\subsection{Bias-variance trade-off}

\subsection{Overfitting}

\subsection{Model selection}

% ===================
% Methods
% ===================
\section{Linear methods}

\subsection{Logistic regression}
% 

\subsection{Support vector machine}
% method
% multi class svm

\section{Tree-based methods}

\subsection{Decision tree and random forest}

\subsection{Extremely randomized trees}

\section{Deep learning}

\subsection{From a single perceptron to a convolutional neural network}

\subsection{Neural network optimization}
% backprop, optimizer, scheduling, learning rate

\subsection{Modern network architectures}
% resnet 
% densenet
% transformers
% unet

\subsection{Deep transfer learning}

\subsection{Multi-task learning}

