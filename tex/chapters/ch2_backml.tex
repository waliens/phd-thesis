\chapter{Machine learning}
\label{chap:backml}

\section{Introduction}
\label{sec:backml:intro}

\acrlong{ml} has been through quite a journey since its inception in the 1940s and 1950s. From a small research domain, it has grown into a massive rapidly-developing field of research and applications with ramifications in many branches of science and technology. This growth is not surprising in a world where computers become more and more powerful and data, the bread-and-butter of machine learning, is becoming increasingly collectable, structured, available and queryable. Applications of \acrshort{ml} are as varied as they are numerous: spam filtering, fraud detection, face recognition, self-driving cars, robotics and automation, medical data analysis and diagnosis, simulation in particle physics... to name only a few. While it has already revolutionized many domains, \acrshort{ml} research has still a bright future ahead. In the last decade only, several new family of techniques have been (re-)discovered and have enabled unexpectedly-fast progress (\eg \acrlong{cnn}, \acrlong{gan}, transformers). One can only expect that the next ground-breaking \acrshort{ml} method is on the brink of being discovered in a research lab somewhere around the world. Aside from new methods, much work has still to be done to actually understand why \acrshort{ml} methods work so well. 

This thesis focuses on the application of machine learning to digital pathology (see Chapter \ref{chap:backdp}) problems. Therefore this chapter aims at giving to a reader with basic knowledge of \acrshort{ml} the keys to understand our contributions. It is not aimed at being a thorough presentation of the different methods but rather an overview. For the readers who would still like to deepen their knowledge about these methods, we will provide pointers to relevant litterature.  

\section{What is machine learning ?} 
\label{sec:backml:whatisml}

\acrlong{ml} is a subfield of artificial intelligence which focuses on methods enabling computers to learn from experience, in the form of data. More formally, a machine learning algorithm is defined by a set of candidate models (\ie an hypothesis space $\mathfrak{H}$), a quality measure for a model and an optimization strategy. The algorithm output is a model $h \in \mathfrak{H}$ that maximises the quality criterion 


\section{}