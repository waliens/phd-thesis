\chapter{Introduction}

Artificial intelligence (\acrshort{ai}) and, more specifically, \textbf{\acrlong{ml}} (\acrshort{ml}) has been through quite a journey since its inception in the 1940s and 1950s. From a small research domain, it has grown into a massive rapidly-developing field of research and applications with ramifications in many branches of science and technology. This growth is not surprising in a world where computers become more and more powerful and data, the bread-and-butter of machine learning, is becoming increasingly collectable, structured, available and queryable. Applications of \acrshort{ml} are as varied as they are numerous: spam filtering, fraud detection, face recognition, self-driving cars, robotics and automation, medical data analysis and diagnosis, simulation in particle physics... to name only a few. While it has already revolutionized many domains, \acrshort{ml} research has still a bright future ahead. In the last decade only, several new family of techniques have been (re-)discovered and have enabled unexpectedly-fast progress (\eg \acrlong{cnn}, \acrlong{gan}, transformers). One can only expect that the next ground-breaking \acrshort{ml} method is on the brink of being discovered in a research lab somewhere around the world. Aside from that, research is still ongoing of many fronts such as understanding, applying or improving existing methods.

Medicine is among the numerous fields where \acrshort{ml} is showing great promises. Although often misrepresented in the mainstream media as a tool that will eventually replace practitioners, the real potential of \acrshort{ml} in medicine actually lies in its capacity to become a strong, resilient and consistent assistant to the experts \parencite{rajkomar2019machine}. Rather than replacing physicians, an \acrshort{ai}-system would be able to complement their opinion and advise them based on experiences of millions of other patients and colleagues. Moreover, such a system would be able to produce these advice based on the whole patient history and a very large number of parameters and sources of data that a human could not realistically consider (imaging, written reports, numerical data such as blood test results, \etc). However, the road to an \acrshort{ai}-assistant is still long and many technological, ethical, legal... questions have to be answered before even considering its potentiality. From a technological point of view, current research focuses on tasks of significantly smaller scale and scope with {\color{red} varying degrees of success} \TODO{reformulate}. For instance,  \parencite{esteva2017dermatologist,gulshan2016development} have shown that \acrshort{ml} methods were able to recognize diseases with as accurately as expert physicians \TODO{check more recent sources}. 

This thesis focuses on the application of machine learning to a field of medical imaging: \textbf{\acrlong{dp}} (\acrshort{dp}). \acrshort{dp} focuses on the analysis of large digitized glass slide images, also called \acrlong{wsi} containing tissue and cell samples