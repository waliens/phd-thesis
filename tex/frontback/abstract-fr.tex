\chapter*{Résumé}

La pathologie, le domaine de la médecine et de la biologie qui s'intéresse à l'étude et au diagnostic des maladies, est sur le point de connaître une révolution grâce aux progrès technologiques de l'intelligence artificielle et de l'apprentissage automatique. Traditionnellement, dans ce domaine, le support utilisé pour la recherche et le diagnostic est une lame de verre sur laquelle sont déposés des échantillons de tissus et de cellules qui sont ensuite analysés au microscope optique. Des scanners spécialisés sont aujourd'hui capables de numériser ces lames de verre en grandes images numériques appelées ``\acrlong{wsi}s'' qui peuvent ensuite être examinées sur un ordinateur. Ce nouveau support offre également une opportunité sans précédent d'utiliser des ordinateurs pour aider les praticiens en automatisant les tâches d'analyse les plus longues et les plus fastidieuses. Le domaine qui s'intéresse à ces questions de numérisation, d'automatisation et aux sujets connexes s'appelle la pathologie digitale.

Les méthodes d'apprentissage automatique et d'apprentissage profond sont d'excellentes candidates pour s'attaquer à ces tâches d'automatisation grâce à leur capacité à apprendre automatiquement des modèles et à capturer des relations complexes directement à partir des données. Cependant, la pathologie numérique présente plusieurs défis pour les méthodes d'apprentissage. En particulier, le domaine souffre d'une pénurie de données, car celles-ci, en particulier annotées, sont difficiles à obtenir en raison de problèmes de confidentialité, du coût des annotations, etc. 

Dans cette thèse, nous explorons différentes techniques d'apprentissage automatique adaptées à la lutte contre la pénurie de données. Nous étudions d'abord différentes techniques d'apprentissage par transfert profond, une famille de méthodes consistant à réutiliser un modèle qui a été appris sur une tâche différente de la tâche cible. Nous étudions les meilleures pratiques concernant la façon dont les modèles neuronaux convolutifs profonds pré-entraînés sur ImageNet, une base de données de photographies, peuvent être transférés à des tâches de classification d'images de pathologie digitale. Nous montrons notamment qu'en pathologie digitale, le ``\textit{fine-tuning}'' surpasse l'extraction de caractéristiques et nous tirons d'autres conclusions pratiques concernant le transfert depuis ImageNet. Motivés par le fait que le transfert est plus performant lorsque les tâches source et cible sont proches, nous utilisons ensuite l'apprentissage multi-tâches pour pré-entraîner un modèle directement sur des données de pathologie. Nous montrons que cette technique est efficace pour créer un modèle transférable adapté aux tâches de pathologie. Enfin, nous abordons le sujet du ``\textit{self-training}'', une famille de méthodes où un modèle en cours d'apprentissage est utilisé pour annoter des données non étiquetées. En particulier, nous appliquons cette technique à la segmentation d'images pour exploiter un ensemble de données qui a été étiqueté de manière lacunaire. Nous montrons que notre approche est capable d'utiliser les données étiquetées lacunairement mieux qu'une approche supervisée. 
