\chapter*{Acknowledgements}

Un doctorat est une aventure parsemée de joies, de peines et d'obstacles. Cette aventure, je n'en aurais jamais vu le bout sans bon nombre de personnes que j'aimerais remercier. 

\vspace{15pt}

Mes premiers remerciements vont évidemment à mes promoteurs, Pierre et Raphaël. Lorsqu'en 2015, j'ai contacté Pierre à la recherche d'un sujet de TFE en ``machine learning sur des images'' et qu'il m'a mis en contact avec Raphaël, je n'aurais jamais osé m'imaginer que cela m'amènerait environ 7 ans plus tard à défendre une thèse de doctorat. Je me souviens encore très bien de la première réunion avec Raphaël à l'extérieur de Montefiore et l'introduction à Cytomine puis les réunions de suivi avec le duo au GIGA. Ces premiers échanges furent aussi enrichissants qu'agréables.

Je me souviens également de la discussion avec Pierre au cours de laquelle il m'a proposé d'entamer une thèse. Si le challenge avait attisé ma curiosité, ce sont aussi les qualités humaines de mes futurs promoteurs qui m'ont convaincues d'accepter la proposition. On entend souvent parler de doctorats qui tournent mal à cause d'une relation doctorant-promoteur compliquée, voire toxique. Dans mon cas, ce fut tout le contraire. J'ai eu l'infinie chance de tomber sur deux personnes bienveillantes, encourageantes, curieuses de ce que j'avais à proposer, toujours prêtes à accorder du temps pour une réunion ou une discussion. Au-delà des aspects humains, leur apport scientifique, technique et leur complémentarité dans ces deux domaines abordés dans ma thèse m'ont évidemment grandement aidé et m'ont permis d'apprendre énormément. Pierre, Raphaël, je ne vous remercierai probablement jamais assez d'avoir cru en moi et de m'avoir permis d'arriver où je suis maintenant!

\vspace{15pt}

J'aimerais aussi remercier tous les collègues que j'ai eu la chance de côtoyer au long de ces six années. Que ça soit en temps de midi, en LAB Meeting, en conférence ou autres, votre présence et nos discussions étaient toujours bien agréables. Vous avez contribué à me faire aimer mon travail ! Un grand merci à Laurine, Jean-Michel, Ulysse, Vân Anh, Pierre, Raphaël, Michaël, Antho, Navdeep, Marie, Antoine, Pascal L., Matthia, Nicolas, Rémy, Pascal F., et encore plein d'autres que j'oublie fort probablement (désolé!).

\vspace{15pt}

Je souhaite également remercier les membres du jury internes, Louis et Marc, et externes, Christine Decaestecker et Francesco Ciompi, qui ont pris le temps de lire mon travail en détail et m'ont fourni un feedback précieux. Merci également à Laurine, Fabrice, Michel, Ulysse et Jean-Michel qui ont relu ce manuscrit avant soumission.

\vspace{15pt}

Ce travail n'aurait jamais été possible sans l'accès à des ressources de calcul et j'aimerais également remercier toutes les personnes m'ont permis de près ou de loin d'y avoir accès. Remerciements, donc, à tous les professeurs et équipes de Montefiore qui ont financé le cluster GPU et remerciements particuliers à Joeri qui a mis en place l'infrastructure.  

\vspace{15pt} 

Amis et famille, je ne vous oublie pas ! Durant ces six années remplies de hauts et de bas, au fond de moi, je n'ai jamais perdu cette petite flamme alimentée par votre amour, amitié et soutien. Quoiqu'il se passe, quelle que soit l'épreuve, je sais que je peux compter sur vous et mine de rien, ça fait toute la différence. 

À ma famille, Papa, Maman et Odile en tête mais aussi tous les autres, merci pour votre présence et votre soutien inconditionnel. Un remerciement tout particulier à Babone et Papy Linlin de m'avoir accueilli tous les vendredis pendant presque un an. Chez vous, j'ai trouvé un havre de quiétude et de concentration qui m'ont permis d'avancer significativement dans la rédaction de ce manuscrit. Un grand merci à toi aussi Florence, qui m'a rejoint sur la fin du parcours pour ton soutien et ton amour pendant la finalisation du travail. 

Les amis, ensuite, que vous soyez présent physiquement ou derrière un écran et un micro, vous m'avez aidé plus que vous ne le pensez dans les périodes difficiles (pandémie, je te vois). Les Componions évidemment (Fabrice, Chloé, Élo, Arnaud, Laurine, Quentin, Hubert), merci pour votre présence, merci d'être ce que vous êtes et de ce que nous devenons. Floriane, merci pour tes visites lors de passages à Montefiore et nos discussions techniques ou non. Michel, merci de m'avoir fait visiter le service d'anatomopathologie du CHU de Liège. Sans toi, mon chapitre 3 ne serait pas ce qu'il est. Les colocs en commençant par Charly, Mirabelle et Nicolas. Que ce fut enrichissant d'emménager et de vivre ensemble. Je n'oublie pas ceux qui ont suivi aussi: Hubert, Michaël, Jérôme, Jiyeon. L'amitié n'est pas toujours là où on l'attend et peut parfois se développer par écran interposé: je tiens à vous remercier les Bloude et en particulier la team "Off" pour tous les bons moments passés ensemble en ligne qui m'ont grandement aidé à passer le cap de la pandémie.

\vspace{15pt}

J'aimerais clôturer ces remerciements avec une mention toute particulière pour toi, Nonno, qui nous a quittés voilà maintenant 11 ans. Tu avais 25 ans et ne savais ni lire ni écrire quand tu as quitté ton Italie natale pour trouver un travail afin de demander la main de Nonna. Tu as échangé le soleil et la chaleur des Pouilles pour le ciel gris belge, la pénombre et la poussière des mines. Ce sont les combats que vous avez menés avec Nonna pour permettre à vos enfants de vivre une vie décente qui m'ont permis d'obtenir le titre de Docteur aujourd'hui. C'est pour cela que je te dédie cette réussite. 
