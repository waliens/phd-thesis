\chapter*{Acknowledgements}

Un doctorat est une aventure parsemée de joies, de peines et d'obstacles. Cette aventure, je n'en aurais jamais vu le bout sans bon nombre de personnes que j'aimerais remercier. 

\vspace{15pt}

Mes permiers remerciements vont évidemment à mes promoteurs, Pierre et Raphaël. Lorsqu'en 2015, j'ai contacté Pierre à la recherche d'un sujet de TFE en ``machine learning sur des images'' et qu'il m'a mis en contact avec Raphaël, je n'aurais jamais osé m'imaginer que cela m'amènerait envion 7 ans plus tard à défendre un thèse de doctorat. Je me souviens encore très bien de la première réunion avec Raphaël à l'extérieur de Montefiore et l'introduction à Cytomine puis les réunions de suivi avec le duo au GIGA notamment, etc. Ces premiers échanges furent aussi enrichissants qu'agréables.

Je me souviens également de la discussion avec Pierre au cours de laquelle il m'a proposé d'entamer une thèse. Si le challenge avait attisé ma curiosité, ce sont aussi les qualités humaines de mes futurs promoteurs qui m'ont convaincues d'accepter la proposition. On entend souvent parler de doctorats qui tournent mal à cause d'une relation doctorant-promoteur compliquée, voire toxique. Dans mon cas, ce fut tout le contraire. J'ai eu l'infinie chance de tomber sur deux personnes bienveillantes, encourageantes, curieuses de ce que j'avais à proposer, toujours prêtent à accorder du temps pour une réunion ou une discussion. Au delà des aspects humains, leur apport scientifique, technique et leur complémentarité dans ces deux domaines abordés par mon sujet de doctorat m'ont évidemment grandement aidé et m'ont permis d'apprendre énormément. Pierre, Raphaël, je ne vous remercierai probablement jamais assez d'avoir cru en moi et de m'avoir permis d'arriver où je suis maintenant!

\vspace{15pt}

J'aimerais aussi remercier tous les collègues que j'ai eu la chance de cotoyer au long de ces six années. Que ça soit en temps de midi, en LAB Meeting, en conférence ou autre, votre présence et nos discussions étaient toujours bien agréables. Vous avez contribué à me faire aimer mon travail ! Un grand merci à Laurine, Jean-Michel, Ulysse, Vân Anh, Pierre, Raphaël, Michaël, Antho, Navdeep, Marie, Antoine, Pascal L., Matthia, Nicolas, Rémy, Pascal F., et encore plein d'autres que j'oublie fort probablement (désolé!).

\vspace{15pt}

Je souhaite également remercier les membres du jury internes, Louis et Marc, et externes, Christine Decaestecker et Francesco Ciompi, qui ont pris le temps de lire mon travail en détail et m'ont fourni un feedback précieux.

\vspace{15pt} 

Amis et famille, je ne vous oublie pas ! Durant ces six années remplies de hauts et de bas, au fond de moi, je n'ai jamais perdu cette petite flamme alimentée par votre amour, amitié et/ou soutien. Quoi qu'il se passe, quelque soit l'épreuve, je sais que je peux compter sur vous et mine de rien, ça fait toute la différence. Les amis, d'abord, que vous soyez présent physiquement ou derrière un écran et un micro, vous m'avez aidé plus que vous ne le pensez dans les périodes difficiles (pandémie, je te vois). Les Compoignons évidemment (Fabrice, Chloé, Élo, Arnaud, Laurine, Quentin, Hubert) merci pour votre présence, merci d'être ce que vous êtes et de ce que nous devenons. Les collocs en commençant par Charly, Mirabelle et Nicolas. Quelle aventure enrichissante ce fut d'emménager et de vivre ensemble. Je n'oublie pas ceux qui ont suivi aussi: Hubert, Michaël, Jérôme, Jiyeon. 


\paragraph{}
babs
None


% Jury 
% Pierre Raphael

% Collègues 
% Jean Michel, Laurine

% Gilles, Joeri

% Amis

% Famille
% parents soeur
% mes grands parents